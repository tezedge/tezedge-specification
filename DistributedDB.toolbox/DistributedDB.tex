\batchmode %% Suppresses most terminal output.
\documentclass{article}
\usepackage{color}
\definecolor{boxshade}{gray}{0.85}
\setlength{\textwidth}{360pt}
\setlength{\textheight}{541pt}
\usepackage{latexsym}
\usepackage{ifthen}
% \usepackage{color}
%%%%%%%%%%%%%%%%%%%%%%%%%%%%%%%%%%%%%%%%%%%%%%%%%%%%%%%%%%%%%%%%%%%%%%%%%%%%%
% SWITCHES                                                                  %
%%%%%%%%%%%%%%%%%%%%%%%%%%%%%%%%%%%%%%%%%%%%%%%%%%%%%%%%%%%%%%%%%%%%%%%%%%%%%
\newboolean{shading} 
\setboolean{shading}{false}
\makeatletter
 %% this is needed only when inserted into the file, not when
 %% used as a package file.
%%%%%%%%%%%%%%%%%%%%%%%%%%%%%%%%%%%%%%%%%%%%%%%%%%%%%%%%%%%%%%%%%%%%%%%%%%%%%
%                                                                           %
% DEFINITIONS OF SYMBOL-PRODUCING COMMANDS                                  %
%                                                                           %
%    TLA+      LaTeX                                                        %
%    symbol    command                                                      %
%    ------    -------                                                      %
%    =>        \implies                                                     %
%    <:        \ltcolon                                                     %
%    :>        \colongt                                                     %
%    ==        \defeq                                                       %
%    ..        \dotdot                                                      %
%    ::        \coloncolon                                                  %
%    =|        \eqdash                                                      %
%    ++        \pp                                                          %
%    --        \mm                                                          %
%    **        \stst                                                        %
%    //        \slsl                                                        %
%    ^         \ct                                                          %
%    \A        \A                                                           %
%    \E        \E                                                           %
%    \AA       \AA                                                          %
%    \EE       \EE                                                          %
%%%%%%%%%%%%%%%%%%%%%%%%%%%%%%%%%%%%%%%%%%%%%%%%%%%%%%%%%%%%%%%%%%%%%%%%%%%%%
\newlength{\symlength}
\newcommand{\implies}{\Rightarrow}
\newcommand{\ltcolon}{\mathrel{<\!\!\mbox{:}}}
\newcommand{\colongt}{\mathrel{\!\mbox{:}\!\!>}}
\newcommand{\defeq}{\;\mathrel{\smash   %% keep this symbol from being too tall
    {{\stackrel{\scriptscriptstyle\Delta}{=}}}}\;}
\newcommand{\dotdot}{\mathrel{\ldotp\ldotp}}
\newcommand{\coloncolon}{\mathrel{::\;}}
\newcommand{\eqdash}{\mathrel = \joinrel \hspace{-.28em}|}
\newcommand{\pp}{\mathbin{++}}
\newcommand{\mm}{\mathbin{--}}
\newcommand{\stst}{*\!*}
\newcommand{\slsl}{/\!/}
\newcommand{\ct}{\hat{\hspace{.4em}}}
\newcommand{\A}{\forall}
\newcommand{\E}{\exists}
\renewcommand{\AA}{\makebox{$\raisebox{.05em}{\makebox[0pt][l]{%
   $\forall\hspace{-.517em}\forall\hspace{-.517em}\forall$}}%
   \forall\hspace{-.517em}\forall \hspace{-.517em}\forall\,$}}
\newcommand{\EE}{\makebox{$\raisebox{.05em}{\makebox[0pt][l]{%
   $\exists\hspace{-.517em}\exists\hspace{-.517em}\exists$}}%
   \exists\hspace{-.517em}\exists\hspace{-.517em}\exists\,$}}
\newcommand{\whileop}{\.{\stackrel
  {\mbox{\raisebox{-.3em}[0pt][0pt]{$\scriptscriptstyle+\;\,$}}}%
  {-\hspace{-.16em}\triangleright}}}

% Commands are defined to produce the upper-case keywords.
% Note that some have space after them.
\newcommand{\ASSUME}{\textsc{assume }}
\newcommand{\ASSUMPTION}{\textsc{assumption }}
\newcommand{\AXIOM}{\textsc{axiom }}
\newcommand{\BOOLEAN}{\textsc{boolean }}
\newcommand{\CASE}{\textsc{case }}
\newcommand{\CONSTANT}{\textsc{constant }}
\newcommand{\CONSTANTS}{\textsc{constants }}
\newcommand{\ELSE}{\settowidth{\symlength}{\THEN}%
   \makebox[\symlength][l]{\textsc{ else}}}
\newcommand{\EXCEPT}{\textsc{ except }}
\newcommand{\EXTENDS}{\textsc{extends }}
\newcommand{\FALSE}{\textsc{false}}
\newcommand{\IF}{\textsc{if }}
\newcommand{\IN}{\settowidth{\symlength}{\LET}%
   \makebox[\symlength][l]{\textsc{in}}}
\newcommand{\INSTANCE}{\textsc{instance }}
\newcommand{\LET}{\textsc{let }}
\newcommand{\LOCAL}{\textsc{local }}
\newcommand{\MODULE}{\textsc{module }}
\newcommand{\OTHER}{\textsc{other }}
\newcommand{\STRING}{\textsc{string}}
\newcommand{\THEN}{\textsc{ then }}
\newcommand{\THEOREM}{\textsc{theorem }}
\newcommand{\LEMMA}{\textsc{lemma }}
\newcommand{\PROPOSITION}{\textsc{proposition }}
\newcommand{\COROLLARY}{\textsc{corollary }}
\newcommand{\TRUE}{\textsc{true}}
\newcommand{\VARIABLE}{\textsc{variable }}
\newcommand{\VARIABLES}{\textsc{variables }}
\newcommand{\WITH}{\textsc{ with }}
\newcommand{\WF}{\textrm{WF}}
\newcommand{\SF}{\textrm{SF}}
\newcommand{\CHOOSE}{\textsc{choose }}
\newcommand{\ENABLED}{\textsc{enabled }}
\newcommand{\UNCHANGED}{\textsc{unchanged }}
\newcommand{\SUBSET}{\textsc{subset }}
\newcommand{\UNION}{\textsc{union }}
\newcommand{\DOMAIN}{\textsc{domain }}
% Added for tla2tex
\newcommand{\BY}{\textsc{by }}
\newcommand{\OBVIOUS}{\textsc{obvious }}
\newcommand{\HAVE}{\textsc{have }}
\newcommand{\QED}{\textsc{qed }}
\newcommand{\TAKE}{\textsc{take }}
\newcommand{\DEF}{\textsc{ def }}
\newcommand{\HIDE}{\textsc{hide }}
\newcommand{\RECURSIVE}{\textsc{recursive }}
\newcommand{\USE}{\textsc{use }}
\newcommand{\DEFINE}{\textsc{define }}
\newcommand{\PROOF}{\textsc{proof }}
\newcommand{\WITNESS}{\textsc{witness }}
\newcommand{\PICK}{\textsc{pick }}
\newcommand{\DEFS}{\textsc{defs }}
\newcommand{\PROVE}{\settowidth{\symlength}{\ASSUME}%
   \makebox[\symlength][l]{\textsc{prove}}\@s{-4.1}}%
  %% The \@s{-4.1) is a kludge added on 24 Oct 2009 [happy birthday, Ellen]
  %% so the correct alignment occurs if the user types
  %%   ASSUME X
  %%   PROVE  Y
  %% because it cancels the extra 4.1 pts added because of the 
  %% extra space after the PROVE.  This seems to works OK.
  %% However, the 4.1 equals Parameters.LaTeXLeftSpace(1) and
  %% should be changed if that method ever changes.
\newcommand{\SUFFICES}{\textsc{suffices }}
\newcommand{\NEW}{\textsc{new }}
\newcommand{\LAMBDA}{\textsc{lambda }}
\newcommand{\STATE}{\textsc{state }}
\newcommand{\ACTION}{\textsc{action }}
\newcommand{\TEMPORAL}{\textsc{temporal }}
\newcommand{\ONLY}{\textsc{only }}              %% added by LL on 2 Oct 2009
\newcommand{\OMITTED}{\textsc{omitted }}        %% added by LL on 31 Oct 2009
\newcommand{\@pfstepnum}[2]{\ensuremath{\langle#1\rangle}\textrm{#2}}
\newcommand{\bang}{\@s{1}\mbox{\small !}\@s{1}}
%% We should format || differently in PlusCal code than in TLA+ formulas.
\newcommand{\p@barbar}{\ifpcalsymbols
   \,\,\rule[-.25em]{.075em}{1em}\hspace*{.2em}\rule[-.25em]{.075em}{1em}\,\,%
   \else \,||\,\fi}
%% PlusCal keywords
\newcommand{\p@fair}{\textbf{fair }}
\newcommand{\p@semicolon}{\textbf{\,; }}
\newcommand{\p@algorithm}{\textbf{algorithm }}
\newcommand{\p@mmfair}{\textbf{-{}-fair }}
\newcommand{\p@mmalgorithm}{\textbf{-{}-algorithm }}
\newcommand{\p@assert}{\textbf{assert }}
\newcommand{\p@await}{\textbf{await }}
\newcommand{\p@begin}{\textbf{begin }}
\newcommand{\p@end}{\textbf{end }}
\newcommand{\p@call}{\textbf{call }}
\newcommand{\p@define}{\textbf{define }}
\newcommand{\p@do}{\textbf{ do }}
\newcommand{\p@either}{\textbf{either }}
\newcommand{\p@or}{\textbf{or }}
\newcommand{\p@goto}{\textbf{goto }}
\newcommand{\p@if}{\textbf{if }}
\newcommand{\p@then}{\,\,\textbf{then }}
\newcommand{\p@else}{\ifcsyntax \textbf{else } \else \,\,\textbf{else }\fi}
\newcommand{\p@elsif}{\,\,\textbf{elsif }}
\newcommand{\p@macro}{\textbf{macro }}
\newcommand{\p@print}{\textbf{print }}
\newcommand{\p@procedure}{\textbf{procedure }}
\newcommand{\p@process}{\textbf{process }}
\newcommand{\p@return}{\textbf{return}}
\newcommand{\p@skip}{\textbf{skip}}
\newcommand{\p@variable}{\textbf{variable }}
\newcommand{\p@variables}{\textbf{variables }}
\newcommand{\p@while}{\textbf{while }}
\newcommand{\p@when}{\textbf{when }}
\newcommand{\p@with}{\textbf{with }}
\newcommand{\p@lparen}{\textbf{(\,\,}}
\newcommand{\p@rparen}{\textbf{\,\,) }}   
\newcommand{\p@lbrace}{\textbf{\{\,\,}}   
\newcommand{\p@rbrace}{\textbf{\,\,\} }}

%%%%%%%%%%%%%%%%%%%%%%%%%%%%%%%%%%%%%%%%%%%%%%%%%%%%%%%%%
% REDEFINE STANDARD COMMANDS TO MAKE THEM FORMAT BETTER %
%                                                       %
% We redefine \in and \notin                            %
%%%%%%%%%%%%%%%%%%%%%%%%%%%%%%%%%%%%%%%%%%%%%%%%%%%%%%%%%
\renewcommand{\_}{\rule{.4em}{.06em}\hspace{.05em}}
\newlength{\equalswidth}
\let\oldin=\in
\let\oldnotin=\notin
\renewcommand{\in}{%
   {\settowidth{\equalswidth}{$\.{=}$}\makebox[\equalswidth][c]{$\oldin$}}}
\renewcommand{\notin}{%
   {\settowidth{\equalswidth}{$\.{=}$}\makebox[\equalswidth]{$\oldnotin$}}}


%%%%%%%%%%%%%%%%%%%%%%%%%%%%%%%%%%%%%%%%%%%%%%%%%%%%
%                                                  %
% HORIZONTAL BARS:                                 %
%                                                  %
%   \moduleLeftDash    |~~~~~~~~~~                 %
%   \moduleRightDash    ~~~~~~~~~~|                %
%   \midbar            |----------|                %
%   \bottombar         |__________|                %
%%%%%%%%%%%%%%%%%%%%%%%%%%%%%%%%%%%%%%%%%%%%%%%%%%%%
\newlength{\charwidth}\settowidth{\charwidth}{{\small\tt M}}
\newlength{\boxrulewd}\setlength{\boxrulewd}{.4pt}
\newlength{\boxlineht}\setlength{\boxlineht}{.5\baselineskip}
\newcommand{\boxsep}{\charwidth}
\newlength{\boxruleht}\setlength{\boxruleht}{.5ex}
\newlength{\boxruledp}\setlength{\boxruledp}{-\boxruleht}
\addtolength{\boxruledp}{\boxrulewd}
\newcommand{\boxrule}{\leaders\hrule height \boxruleht depth \boxruledp
                      \hfill\mbox{}}
\newcommand{\@computerule}{%
  \setlength{\boxruleht}{.5ex}%
  \setlength{\boxruledp}{-\boxruleht}%
  \addtolength{\boxruledp}{\boxrulewd}}

\newcommand{\bottombar}{\hspace{-\boxsep}%
  \raisebox{-\boxrulewd}[0pt][0pt]{\rule[.5ex]{\boxrulewd}{\boxlineht}}%
  \boxrule
  \raisebox{-\boxrulewd}[0pt][0pt]{%
      \rule[.5ex]{\boxrulewd}{\boxlineht}}\hspace{-\boxsep}\vspace{0pt}}

\newcommand{\moduleLeftDash}%
   {\hspace*{-\boxsep}%
     \raisebox{-\boxlineht}[0pt][0pt]{\rule[.5ex]{\boxrulewd
               }{\boxlineht}}%
    \boxrule\hspace*{.4em }}

\newcommand{\moduleRightDash}%
    {\hspace*{.4em}\boxrule
    \raisebox{-\boxlineht}[0pt][0pt]{\rule[.5ex]{\boxrulewd
               }{\boxlineht}}\hspace{-\boxsep}}%\vspace{.2em}

\newcommand{\midbar}{\hspace{-\boxsep}\raisebox{-.5\boxlineht}[0pt][0pt]{%
   \rule[.5ex]{\boxrulewd}{\boxlineht}}\boxrule\raisebox{-.5\boxlineht%
   }[0pt][0pt]{\rule[.5ex]{\boxrulewd}{\boxlineht}}\hspace{-\boxsep}}

%%%%%%%%%%%%%%%%%%%%%%%%%%%%%%%%%%%%%%%%%%%%%%%%%%%%%%%%%%%%%%%%%%%%%%%%%%%%%
% FORMATING COMMANDS                                                        %
%%%%%%%%%%%%%%%%%%%%%%%%%%%%%%%%%%%%%%%%%%%%%%%%%%%%%%%%%%%%%%%%%%%%%%%%%%%%%

%%%%%%%%%%%%%%%%%%%%%%%%%%%%%%%%%%%%%%%%%%%%%%%%%%%%%%%%%%%%%%%%%%%%%%%%%%%%%
% PLUSCAL SHADING                                                           %
%%%%%%%%%%%%%%%%%%%%%%%%%%%%%%%%%%%%%%%%%%%%%%%%%%%%%%%%%%%%%%%%%%%%%%%%%%%%%

% The TeX pcalshading switch is set on to cause PlusCal shading to be
% performed.  This changes the behavior of the following commands and
% environments to cause full-width shading to be performed on all lines.
% 
%   \tstrut \@x cpar mcom \@pvspace
% 
% The TeX pcalsymbols switch is turned on when typesetting a PlusCal algorithm,
% whether or not shading is being performed.  It causes symbols (other than
% parentheses and braces and PlusCal-only keywords) that should be typeset
% differently depending on whether they are in an algorithm to be typeset
% appropriately.  Currently, the only such symbol is "||".
%
% The TeX csyntax switch is turned on when typesetting a PlusCal algorithm in
% c-syntax.  This allows symbols to be format differently in the two syntaxes.
% The "else" keyword is the only one that is.

\newif\ifpcalshading \pcalshadingfalse
\newif\ifpcalsymbols \pcalsymbolsfalse
\newif\ifcsyntax     \csyntaxtrue

% The \@pvspace command makes a vertical space.  It uses \vspace
% except with \ifpcalshading, in which case it sets \pvcalvspace
% and the space is added by a following \@x command.
%
\newlength{\pcalvspace}\setlength{\pcalvspace}{0pt}%
\newcommand{\@pvspace}[1]{%
  \ifpcalshading
     \par\global\setlength{\pcalvspace}{#1}%
  \else
     \par\vspace{#1}%
  \fi
}

% The lcom environment was changed to set \lcomindent equal to
% the indentation it produces.  This length is used by the
% cpar environment to make shading extend for the full width
% of the line.  This assumes that lcom environments are not
% nested.  I hope TLATeX does not nest them.
%
\newlength{\lcomindent}%
\setlength{\lcomindent}{0pt}%

%\tstrut: A strut to produce inter-paragraph space in a comment.
%\rstrut: A strut to extend the bottom of a one-line comment so
%         there's no break in the shading between comments on 
%         successive lines.
\newcommand\tstrut%
  {\raisebox{\vshadelen}{\raisebox{-.25em}{\rule{0pt}{1.15em}}}%
   \global\setlength{\vshadelen}{0pt}}
\newcommand\rstrut{\raisebox{-.25em}{\rule{0pt}{1.15em}}%
 \global\setlength{\vshadelen}{0pt}}


% \.{op} formats operator op in math mode with empty boxes on either side.
% Used because TeX otherwise vary the amount of space it leaves around op.
\renewcommand{\.}[1]{\ensuremath{\mbox{}#1\mbox{}}}

% \@s{n} produces an n-point space
\newcommand{\@s}[1]{\hspace{#1pt}}           

% \@x{txt} starts a specification line in the beginning with txt
% in the final LaTeX source.
\newlength{\@xlen}
\newcommand\xtstrut%
  {\setlength{\@xlen}{1.05em}%
   \addtolength{\@xlen}{\pcalvspace}%
    \raisebox{\vshadelen}{\raisebox{-.25em}{\rule{0pt}{\@xlen}}}%
   \global\setlength{\vshadelen}{0pt}%
   \global\setlength{\pcalvspace}{0pt}}

\newcommand{\@x}[1]{\par
  \ifpcalshading
  \makebox[0pt][l]{\shadebox{\xtstrut\hspace*{\textwidth}}}%
  \fi
  \mbox{$\mbox{}#1\mbox{}$}}  

% \@xx{txt} continues a specification line with the text txt.
\newcommand{\@xx}[1]{\mbox{$\mbox{}#1\mbox{}$}}  

% \@y{cmt} produces a one-line comment.
\newcommand{\@y}[1]{\mbox{\footnotesize\hspace{.65em}%
  \ifthenelse{\boolean{shading}}{%
      \shadebox{#1\hspace{-\the\lastskip}\rstrut}}%
               {#1\hspace{-\the\lastskip}\rstrut}}}

% \@z{cmt} produces a zero-width one-line comment.
\newcommand{\@z}[1]{\makebox[0pt][l]{\footnotesize
  \ifthenelse{\boolean{shading}}{%
      \shadebox{#1\hspace{-\the\lastskip}\rstrut}}%
               {#1\hspace{-\the\lastskip}\rstrut}}}


% \@w{str} produces the TLA+ string "str".
\newcommand{\@w}[1]{\textsf{``{#1}''}}             


%%%%%%%%%%%%%%%%%%%%%%%%%%%%%%%%%%%%%%%%%%%%%%%%%%%%%%%%%%%%%%%%%%%%%%%%%%%%%
% SHADING                                                                   %
%%%%%%%%%%%%%%%%%%%%%%%%%%%%%%%%%%%%%%%%%%%%%%%%%%%%%%%%%%%%%%%%%%%%%%%%%%%%%
\def\graymargin{1}
  % The number of points of margin in the shaded box.

% \definecolor{boxshade}{gray}{.85}
% Defines the darkness of the shading: 1 = white, 0 = black
% Added by TLATeX only if needed.

% \shadebox{txt} puts txt in a shaded box.
\newlength{\templena}
\newlength{\templenb}
\newsavebox{\tempboxa}
\newcommand{\shadebox}[1]{{\setlength{\fboxsep}{\graymargin pt}%
     \savebox{\tempboxa}{#1}%
     \settoheight{\templena}{\usebox{\tempboxa}}%
     \settodepth{\templenb}{\usebox{\tempboxa}}%
     \hspace*{-\fboxsep}\raisebox{0pt}[\templena][\templenb]%
        {\colorbox{boxshade}{\usebox{\tempboxa}}}\hspace*{-\fboxsep}}}

% \vshade{n} makes an n-point inter-paragraph space, with
%  shading if the `shading' flag is true.
\newlength{\vshadelen}
\setlength{\vshadelen}{0pt}
\newcommand{\vshade}[1]{\ifthenelse{\boolean{shading}}%
   {\global\setlength{\vshadelen}{#1pt}}%
   {\vspace{#1pt}}}

\newlength{\boxwidth}
\newlength{\multicommentdepth}

%%%%%%%%%%%%%%%%%%%%%%%%%%%%%%%%%%%%%%%%%%%%%%%%%%%%%%%%%%%%%%%%%%%%%%%%%%%%%
% THE cpar ENVIRONMENT                                                      %
% ^^^^^^^^^^^^^^^^^^^^                                                      %
% The LaTeX input                                                           %
%                                                                           %
%   \begin{cpar}{pop}{nest}{isLabel}{d}{e}{arg6}                            %
%     XXXXXXXXXXXXXXX                                                       %
%     XXXXXXXXXXXXXXX                                                       %
%     XXXXXXXXXXXXXXX                                                       %
%   \end{cpar}                                                              %
%                                                                           %
% produces one of two possible results.  If isLabel is the letter "T",      %
% it produces the following, where [label] is the result of typesetting     %
% arg6 in an LR box, and d is is a number representing a distance in        %
% points.                                                                   %
%                                                                           %
%   prevailing |<-- d -->[label]<- e ->XXXXXXXXXXXXXXX                      %
%         left |                       XXXXXXXXXXXXXXX                      %
%       margin |                       XXXXXXXXXXXXXXX                      %
%                                                                           %
% If isLabel is the letter "F", then it produces                            %
%                                                                           %
%   prevailing |<-- d -->XXXXXXXXXXXXXXXXXXXXXXX                            %
%         left |         <- e ->XXXXXXXXXXXXXXXX                            %
%       margin |                XXXXXXXXXXXXXXXX                            %
%                                                                           %
% where d and e are numbers representing distances in points.               %
%                                                                           %
% The prevailing left margin is the one in effect before the most recent    %
% pop (argument 1) cpar environments with "T" as the nest argument, where   %
% pop is a number \geq 0.                                                   %
%                                                                           %
% If the nest argument is the letter "T", then the prevailing left          %
% margin is moved to the left of the second (and following) lines of        %
% X's.  Otherwise, the prevailing left margin is left unchanged.            %
%                                                                           %
% An \unnest{n} command moves the prevailing left margin to where it was    %
% before the most recent n cpar environments with "T" as the nesting        %
% argument.                                                                 %
%                                                                           %
% The environment leaves no vertical space above or below it, or between    %
% its paragraphs.  (TLATeX inserts the proper amount of vertical space.)    %
%%%%%%%%%%%%%%%%%%%%%%%%%%%%%%%%%%%%%%%%%%%%%%%%%%%%%%%%%%%%%%%%%%%%%%%%%%%%%

\newcounter{pardepth}
\setcounter{pardepth}{0}

% \setgmargin{txt} defines \gmarginN to be txt, where N is \roman{pardepth}.
% \thegmargin equals \gmarginN, where N is \roman{pardepth}.
\newcommand{\setgmargin}[1]{%
  \expandafter\xdef\csname gmargin\roman{pardepth}\endcsname{#1}}
\newcommand{\thegmargin}{\csname gmargin\roman{pardepth}\endcsname}
\newcommand{\gmargin}{0pt}

\newsavebox{\tempsbox}

\newlength{\@cparht}
\newlength{\@cpardp}
\newenvironment{cpar}[6]{%
  \addtocounter{pardepth}{-#1}%
  \ifthenelse{\boolean{shading}}{\par\begin{lrbox}{\tempsbox}%
                                 \begin{minipage}[t]{\linewidth}}{}%
  \begin{list}{}{%
     \edef\temp{\thegmargin}
     \ifthenelse{\equal{#3}{T}}%
       {\settowidth{\leftmargin}{\hspace{\temp}\footnotesize #6\hspace{#5pt}}%
        \addtolength{\leftmargin}{#4pt}}%
       {\setlength{\leftmargin}{#4pt}%
        \addtolength{\leftmargin}{#5pt}%
        \addtolength{\leftmargin}{\temp}%
        \setlength{\itemindent}{-#5pt}}%
      \ifthenelse{\equal{#2}{T}}{\addtocounter{pardepth}{1}%
                                 \setgmargin{\the\leftmargin}}{}%
      \setlength{\labelwidth}{0pt}%
      \setlength{\labelsep}{0pt}%
      \setlength{\itemindent}{-\leftmargin}%
      \setlength{\topsep}{0pt}%
      \setlength{\parsep}{0pt}%
      \setlength{\partopsep}{0pt}%
      \setlength{\parskip}{0pt}%
      \setlength{\itemsep}{0pt}
      \setlength{\itemindent}{#4pt}%
      \addtolength{\itemindent}{-\leftmargin}}%
   \ifthenelse{\equal{#3}{T}}%
      {\item[\tstrut\footnotesize \hspace{\temp}{#6}\hspace{#5pt}]
        }%
      {\item[\tstrut\hspace{\temp}]%
         }%
   \footnotesize}
 {\hspace{-\the\lastskip}\tstrut
 \end{list}%
  \ifthenelse{\boolean{shading}}%
          {\end{minipage}%
           \end{lrbox}%
           \ifpcalshading
             \setlength{\@cparht}{\ht\tempsbox}%
             \setlength{\@cpardp}{\dp\tempsbox}%
             \addtolength{\@cparht}{.15em}%
             \addtolength{\@cpardp}{.2em}%
             \addtolength{\@cparht}{\@cpardp}%
            % I don't know what's going on here.  I want to add a
            % \pcalvspace high shaded line, but I don't know how to
            % do it.  A little trial and error shows that the following
            % does a reasonable job approximating that, eliminating
            % the line if \pcalvspace is small.
            \addtolength{\@cparht}{\pcalvspace}%
             \ifdim \pcalvspace > .8em
               \addtolength{\pcalvspace}{-.2em}%
               \hspace*{-\lcomindent}%
               \shadebox{\rule{0pt}{\pcalvspace}\hspace*{\textwidth}}\par
               \global\setlength{\pcalvspace}{0pt}%
               \fi
             \hspace*{-\lcomindent}%
             \makebox[0pt][l]{\raisebox{-\@cpardp}[0pt][0pt]{%
                 \shadebox{\rule{0pt}{\@cparht}\hspace*{\textwidth}}}}%
             \hspace*{\lcomindent}\usebox{\tempsbox}%
             \par
           \else
             \shadebox{\usebox{\tempsbox}}\par
           \fi}%
           {}%
  }

%%%%%%%%%%%%%%%%%%%%%%%%%%%%%%%%%%%%%%%%%%%%%%%%%%%%%%%%%%%%%%%%%%%%%%%%%%%%%%
% THE ppar ENVIRONMENT                                                       %
% ^^^^^^^^^^^^^^^^^^^^                                                       %
% The environment                                                            %
%                                                                            %
%   \begin{ppar} ... \end{ppar}                                              %
%                                                                            %
% is equivalent to                                                           %
%                                                                            %
%   \begin{cpar}{0}{F}{F}{0}{0}{} ... \end{cpar}                             %
%                                                                            %
% The environment is put around each line of the output for a PlusCal        %
% algorithm.                                                                 %
%%%%%%%%%%%%%%%%%%%%%%%%%%%%%%%%%%%%%%%%%%%%%%%%%%%%%%%%%%%%%%%%%%%%%%%%%%%%%%
%\newenvironment{ppar}{%
%  \ifthenelse{\boolean{shading}}{\par\begin{lrbox}{\tempsbox}%
%                                 \begin{minipage}[t]{\linewidth}}{}%
%  \begin{list}{}{%
%     \edef\temp{\thegmargin}
%        \setlength{\leftmargin}{0pt}%
%        \addtolength{\leftmargin}{\temp}%
%        \setlength{\itemindent}{0pt}%
%      \setlength{\labelwidth}{0pt}%
%      \setlength{\labelsep}{0pt}%
%      \setlength{\itemindent}{-\leftmargin}%
%      \setlength{\topsep}{0pt}%
%      \setlength{\parsep}{0pt}%
%      \setlength{\partopsep}{0pt}%
%      \setlength{\parskip}{0pt}%
%      \setlength{\itemsep}{0pt}
%      \setlength{\itemindent}{0pt}%
%      \addtolength{\itemindent}{-\leftmargin}}%
%      \item[\tstrut\hspace{\temp}]}%
% {\hspace{-\the\lastskip}\tstrut
% \end{list}%
%  \ifthenelse{\boolean{shading}}{\end{minipage}  
%                                 \end{lrbox}%
%                                 \shadebox{\usebox{\tempsbox}}\par}{}%
%  }

 %%% TESTING
 \newcommand{\xtest}[1]{\par
 \makebox[0pt][l]{\shadebox{\xtstrut\hspace*{\textwidth}}}%
 \mbox{$\mbox{}#1\mbox{}$}} 

% \newcommand{\xxtest}[1]{\par
% \makebox[0pt][l]{\shadebox{\xtstrut{#1}\hspace*{\textwidth}}}%
% \mbox{$\mbox{}#1\mbox{}$}} 

%\newlength{\pcalvspace}
%\setlength{\pcalvspace}{0pt}
% \newlength{\xxtestlen}
% \setlength{\xxtestlen}{0pt}
% \newcommand\xtstrut%
%   {\setlength{\xxtestlen}{1.15em}%
%    \addtolength{\xxtestlen}{\pcalvspace}%
%     \raisebox{\vshadelen}{\raisebox{-.25em}{\rule{0pt}{\xxtestlen}}}%
%    \global\setlength{\vshadelen}{0pt}%
%    \global\setlength{\pcalvspace}{0pt}}
   
   %%%% TESTING
   
   %% The xcpar environment
   %%  Note: overloaded use of \pcalvspace for testing.
   %%
%   \newlength{\xcparht}%
%   \newlength{\xcpardp}%
   
%   \newenvironment{xcpar}[6]{%
%  \addtocounter{pardepth}{-#1}%
%  \ifthenelse{\boolean{shading}}{\par\begin{lrbox}{\tempsbox}%
%                                 \begin{minipage}[t]{\linewidth}}{}%
%  \begin{list}{}{%
%     \edef\temp{\thegmargin}%
%     \ifthenelse{\equal{#3}{T}}%
%       {\settowidth{\leftmargin}{\hspace{\temp}\footnotesize #6\hspace{#5pt}}%
%        \addtolength{\leftmargin}{#4pt}}%
%       {\setlength{\leftmargin}{#4pt}%
%        \addtolength{\leftmargin}{#5pt}%
%        \addtolength{\leftmargin}{\temp}%
%        \setlength{\itemindent}{-#5pt}}%
%      \ifthenelse{\equal{#2}{T}}{\addtocounter{pardepth}{1}%
%                                 \setgmargin{\the\leftmargin}}{}%
%      \setlength{\labelwidth}{0pt}%
%      \setlength{\labelsep}{0pt}%
%      \setlength{\itemindent}{-\leftmargin}%
%      \setlength{\topsep}{0pt}%
%      \setlength{\parsep}{0pt}%
%      \setlength{\partopsep}{0pt}%
%      \setlength{\parskip}{0pt}%
%      \setlength{\itemsep}{0pt}%
%      \setlength{\itemindent}{#4pt}%
%      \addtolength{\itemindent}{-\leftmargin}}%
%   \ifthenelse{\equal{#3}{T}}%
%      {\item[\xtstrut\footnotesize \hspace{\temp}{#6}\hspace{#5pt}]%
%        }%
%      {\item[\xtstrut\hspace{\temp}]%
%         }%
%   \footnotesize}
% {\hspace{-\the\lastskip}\tstrut
% \end{list}%
%  \ifthenelse{\boolean{shading}}{\end{minipage}  
%                                 \end{lrbox}%
%                                 \setlength{\xcparht}{\ht\tempsbox}%
%                                 \setlength{\xcpardp}{\dp\tempsbox}%
%                                 \addtolength{\xcparht}{.15em}%
%                                 \addtolength{\xcpardp}{.2em}%
%                                 \addtolength{\xcparht}{\xcpardp}%
%                                 \hspace*{-\lcomindent}%
%                                 \makebox[0pt][l]{\raisebox{-\xcpardp}[0pt][0pt]{%
%                                      \shadebox{\rule{0pt}{\xcparht}\hspace*{\textwidth}}}}%
%                                 \hspace*{\lcomindent}\usebox{\tempsbox}%
%                                 \par}{}%
%  }
%  
% \newlength{\xmcomlen}
%\newenvironment{xmcom}[1]{%
%  \setcounter{pardepth}{0}%
%  \hspace{.65em}%
%  \begin{lrbox}{\alignbox}\sloppypar%
%      \setboolean{shading}{false}%
%      \setlength{\boxwidth}{#1pt}%
%      \addtolength{\boxwidth}{-.65em}%
%      \begin{minipage}[t]{\boxwidth}\footnotesize
%      \parskip=0pt\relax}%
%       {\end{minipage}\end{lrbox}%
%       \setlength{\xmcomlen}{\textwidth}%
%       \addtolength{\xmcomlen}{-\wd\alignbox}%
%       \settodepth{\alignwidth}{\usebox{\alignbox}}%
%       \global\setlength{\multicommentdepth}{\alignwidth}%
%       \setlength{\boxwidth}{\alignwidth}%
%       \global\addtolength{\alignwidth}{-\maxdepth}%
%       \addtolength{\boxwidth}{.1em}%
%       \raisebox{0pt}[0pt][0pt]{%
%        \ifthenelse{\boolean{shading}}%
%          {\hspace*{-\xmcomlen}\shadebox{\rule[-\boxwidth]{0pt}{0pt}%
%                                 \hspace*{\xmcomlen}\usebox{\alignbox}}}%
%          {\usebox{\alignbox}}}%
%       \vspace*{\alignwidth}\pagebreak[0]\vspace{-\alignwidth}\par}
% % a multi-line comment, whose first argument is its width in points.
%  
   
%%%%%%%%%%%%%%%%%%%%%%%%%%%%%%%%%%%%%%%%%%%%%%%%%%%%%%%%%%%%%%%%%%%%%%%%%%%%%%
% THE lcom ENVIRONMENT                                                       %
% ^^^^^^^^^^^^^^^^^^^^                                                       %
% A multi-line comment with no text to its left is typeset in an lcom        % 
% environment, whose argument is a number representing the indentation       % 
% of the left margin, in points.  All the text of the comment should be      % 
% inside cpar environments.                                                  % 
%%%%%%%%%%%%%%%%%%%%%%%%%%%%%%%%%%%%%%%%%%%%%%%%%%%%%%%%%%%%%%%%%%%%%%%%%%%%%%
\newenvironment{lcom}[1]{%
  \setlength{\lcomindent}{#1pt} % Added for PlusCal handling.
  \par\vspace{.2em}%
  \sloppypar
  \setcounter{pardepth}{0}%
  \footnotesize
  \begin{list}{}{%
    \setlength{\leftmargin}{#1pt}
    \setlength{\labelwidth}{0pt}%
    \setlength{\labelsep}{0pt}%
    \setlength{\itemindent}{0pt}%
    \setlength{\topsep}{0pt}%
    \setlength{\parsep}{0pt}%
    \setlength{\partopsep}{0pt}%
    \setlength{\parskip}{0pt}}
    \item[]}%
  {\end{list}\vspace{.3em}\setlength{\lcomindent}{0pt}%
 }


%%%%%%%%%%%%%%%%%%%%%%%%%%%%%%%%%%%%%%%%%%%%%%%%%%%%%%%%%%%%%%%%%%%%%%%%%%%%%
% THE mcom ENVIRONMENT AND \mutivspace COMMAND                              %
% ^^^^^^^^^^^^^^^^^^^^^^^^^^^^^^^^^^^^^^^^^^^^                              %
%                                                                           %
% A part of the spec containing a right-comment of the form                 %
%                                                                           %
%      xxxx (*************)                                                 %
%      yyyy (* ccccccccc *)                                                 %
%      ...  (* ccccccccc *)                                                 %
%           (* ccccccccc *)                                                 %
%           (* ccccccccc *)                                                 %
%           (*************)                                                 %
%                                                                           %
% is typeset by                                                             %
%                                                                           %
%     XXXX \begin{mcom}{d}                                                  %
%            CCCC ... CCC                                                   %
%          \end{mcom}                                                       %
%     YYYY ...                                                              %
%     \multivspace{n}                                                       %
%                                                                           %
% where the number d is the width in points of the comment, n is the        %
% number of xxxx, yyyy, ...  lines to the left of the comment.              %
% All the text of the comment should be typeset in cpar environments.       %
%                                                                           %
% This puts the comment into a single box (so no page breaks can occur      %
% within it).  The entire box is shaded iff the shading flag is true.       %
%%%%%%%%%%%%%%%%%%%%%%%%%%%%%%%%%%%%%%%%%%%%%%%%%%%%%%%%%%%%%%%%%%%%%%%%%%%%%
\newlength{\xmcomlen}%
\newenvironment{mcom}[1]{%
  \setcounter{pardepth}{0}%
  \hspace{.65em}%
  \begin{lrbox}{\alignbox}\sloppypar%
      \setboolean{shading}{false}%
      \setlength{\boxwidth}{#1pt}%
      \addtolength{\boxwidth}{-.65em}%
      \begin{minipage}[t]{\boxwidth}\footnotesize
      \parskip=0pt\relax}%
       {\end{minipage}\end{lrbox}%
       \setlength{\xmcomlen}{\textwidth}%       % For PlusCal shading
       \addtolength{\xmcomlen}{-\wd\alignbox}%  % For PlusCal shading
       \settodepth{\alignwidth}{\usebox{\alignbox}}%
       \global\setlength{\multicommentdepth}{\alignwidth}%
       \setlength{\boxwidth}{\alignwidth}%      % For PlusCal shading
       \global\addtolength{\alignwidth}{-\maxdepth}%
       \addtolength{\boxwidth}{.1em}%           % For PlusCal shading
      \raisebox{0pt}[0pt][0pt]{%
        \ifthenelse{\boolean{shading}}%
          {\ifpcalshading
             \hspace*{-\xmcomlen}%
             \shadebox{\rule[-\boxwidth]{0pt}{0pt}\hspace*{\xmcomlen}%
                          \usebox{\alignbox}}%
           \else
             \shadebox{\usebox{\alignbox}}
           \fi
          }%
          {\usebox{\alignbox}}}%
       \vspace*{\alignwidth}\pagebreak[0]\vspace{-\alignwidth}\par}
 % a multi-line comment, whose first argument is its width in points.


% \multispace{n} produces the vertical space indicated by "|"s in 
% this situation
%   
%     xxxx (*************)
%     xxxx (* ccccccccc *)
%      |   (* ccccccccc *)
%      |   (* ccccccccc *)
%      |   (* ccccccccc *)
%      |   (*************)
%
% where n is the number of "xxxx" lines.
\newcommand{\multivspace}[1]{\addtolength{\multicommentdepth}{-#1\baselineskip}%
 \addtolength{\multicommentdepth}{1.2em}%
 \ifthenelse{\lengthtest{\multicommentdepth > 0pt}}%
    {\par\vspace{\multicommentdepth}\par}{}}

%\newenvironment{hpar}[2]{%
%  \begin{list}{}{\setlength{\leftmargin}{#1pt}%
%                 \addtolength{\leftmargin}{#2pt}%
%                 \setlength{\itemindent}{-#2pt}%
%                 \setlength{\topsep}{0pt}%
%                 \setlength{\parsep}{0pt}%
%                 \setlength{\partopsep}{0pt}%
%                 \setlength{\parskip}{0pt}%
%                 \addtolength{\labelsep}{0pt}}%
%  \item[]\footnotesize}{\end{list}}
%    %%%%%%%%%%%%%%%%%%%%%%%%%%%%%%%%%%%%%%%%%%%%%%%%%%%%%%%%%%%%%%%%%%%%%%%%
%    % Typesets a sequence of paragraphs like this:                         %
%    %                                                                      %
%    %      left |<-- d1 --> XXXXXXXXXXXXXXXXXXXXXXXX                       %
%    %    margin |           <- d2 -> XXXXXXXXXXXXXXX                       %
%    %           |                    XXXXXXXXXXXXXXX                       %
%    %           |                                                          %
%    %           |                    XXXXXXXXXXXXXXX                       %
%    %           |                    XXXXXXXXXXXXXXX                       %
%    %                                                                      %
%    % where d1 = #1pt and d2 = #2pt, but with no vspace between            %
%    % paragraphs.                                                          %
%    %%%%%%%%%%%%%%%%%%%%%%%%%%%%%%%%%%%%%%%%%%%%%%%%%%%%%%%%%%%%%%%%%%%%%%%%

%%%%%%%%%%%%%%%%%%%%%%%%%%%%%%%%%%%%%%%%%%%%%%%%%%%%%%%%%%%%%%%%%%%%%%
% Commands for repeated characters that produce dashes.              %
%%%%%%%%%%%%%%%%%%%%%%%%%%%%%%%%%%%%%%%%%%%%%%%%%%%%%%%%%%%%%%%%%%%%%%
% \raisedDash{wd}{ht}{thk} makes a horizontal line wd characters wide, 
% raised a distance ht ex's above the baseline, with a thickness of 
% thk em's.
\newcommand{\raisedDash}[3]{\raisebox{#2ex}{\setlength{\alignwidth}{.5em}%
  \rule{#1\alignwidth}{#3em}}}

% The following commands take a single argument n and produce the
% output for n repeated characters, as follows
%   \cdash:    -
%   \tdash:    ~
%   \ceqdash:  =
%   \usdash:   _
\newcommand{\cdash}[1]{\raisedDash{#1}{.5}{.04}}
\newcommand{\usdash}[1]{\raisedDash{#1}{0}{.04}}
\newcommand{\ceqdash}[1]{\raisedDash{#1}{.5}{.08}}
\newcommand{\tdash}[1]{\raisedDash{#1}{1}{.08}}

\newlength{\spacewidth}
\setlength{\spacewidth}{.2em}
\newcommand{\e}[1]{\hspace{#1\spacewidth}}
%% \e{i} produces space corresponding to i input spaces.


%% Alignment-file Commands

\newlength{\alignboxwidth}
\newlength{\alignwidth}
\newsavebox{\alignbox}

% \al{i}{j}{txt} is used in the alignment file to put "%{i}{j}{wd}"
% in the log file, where wd is the width of the line up to that point,
% and txt is the following text.
\newcommand{\al}[3]{%
  \typeout{\%{#1}{#2}{\the\alignwidth}}%
  \cl{#3}}

%% \cl{txt} continues a specification line in the alignment file
%% with text txt.
\newcommand{\cl}[1]{%
  \savebox{\alignbox}{\mbox{$\mbox{}#1\mbox{}$}}%
  \settowidth{\alignboxwidth}{\usebox{\alignbox}}%
  \addtolength{\alignwidth}{\alignboxwidth}%
  \usebox{\alignbox}}

% \fl{txt} in the alignment file begins a specification line that
% starts with the text txt.
\newcommand{\fl}[1]{%
  \par
  \savebox{\alignbox}{\mbox{$\mbox{}#1\mbox{}$}}%
  \settowidth{\alignwidth}{\usebox{\alignbox}}%
  \usebox{\alignbox}}



  
%%%%%%%%%%%%%%%%%%%%%%%%%%%%%%%%%%%%%%%%%%%%%%%%%%%%%%%%%%%%%%%%%%%%%%%%%%%%%
% Ordinarily, TeX typesets letters in math mode in a special math italic    %
% font.  This makes it typeset "it" to look like the product of the         %
% variables i and t, rather than like the word "it".  The following         %
% commands tell TeX to use an ordinary italic font instead.                 %
%%%%%%%%%%%%%%%%%%%%%%%%%%%%%%%%%%%%%%%%%%%%%%%%%%%%%%%%%%%%%%%%%%%%%%%%%%%%%
\ifx\documentclass\undefined
\else
  \DeclareSymbolFont{tlaitalics}{\encodingdefault}{cmr}{m}{it}
  \let\itfam\symtlaitalics
\fi

\makeatletter
\newcommand{\tlx@c}{\c@tlx@ctr\advance\c@tlx@ctr\@ne}
\newcounter{tlx@ctr}
\c@tlx@ctr=\itfam \multiply\c@tlx@ctr"100\relax \advance\c@tlx@ctr "7061\relax
\mathcode`a=\tlx@c \mathcode`b=\tlx@c \mathcode`c=\tlx@c \mathcode`d=\tlx@c
\mathcode`e=\tlx@c \mathcode`f=\tlx@c \mathcode`g=\tlx@c \mathcode`h=\tlx@c
\mathcode`i=\tlx@c \mathcode`j=\tlx@c \mathcode`k=\tlx@c \mathcode`l=\tlx@c
\mathcode`m=\tlx@c \mathcode`n=\tlx@c \mathcode`o=\tlx@c \mathcode`p=\tlx@c
\mathcode`q=\tlx@c \mathcode`r=\tlx@c \mathcode`s=\tlx@c \mathcode`t=\tlx@c
\mathcode`u=\tlx@c \mathcode`v=\tlx@c \mathcode`w=\tlx@c \mathcode`x=\tlx@c
\mathcode`y=\tlx@c \mathcode`z=\tlx@c
\c@tlx@ctr=\itfam \multiply\c@tlx@ctr"100\relax \advance\c@tlx@ctr "7041\relax
\mathcode`A=\tlx@c \mathcode`B=\tlx@c \mathcode`C=\tlx@c \mathcode`D=\tlx@c
\mathcode`E=\tlx@c \mathcode`F=\tlx@c \mathcode`G=\tlx@c \mathcode`H=\tlx@c
\mathcode`I=\tlx@c \mathcode`J=\tlx@c \mathcode`K=\tlx@c \mathcode`L=\tlx@c
\mathcode`M=\tlx@c \mathcode`N=\tlx@c \mathcode`O=\tlx@c \mathcode`P=\tlx@c
\mathcode`Q=\tlx@c \mathcode`R=\tlx@c \mathcode`S=\tlx@c \mathcode`T=\tlx@c
\mathcode`U=\tlx@c \mathcode`V=\tlx@c \mathcode`W=\tlx@c \mathcode`X=\tlx@c
\mathcode`Y=\tlx@c \mathcode`Z=\tlx@c
\makeatother

%%%%%%%%%%%%%%%%%%%%%%%%%%%%%%%%%%%%%%%%%%%%%%%%%%%%%%%%%%
%                THE describe ENVIRONMENT                %
%%%%%%%%%%%%%%%%%%%%%%%%%%%%%%%%%%%%%%%%%%%%%%%%%%%%%%%%%%
%
%
% It is like the description environment except it takes an argument
% ARG that should be the text of the widest label.  It adjusts the
% indentation so each item with label LABEL produces
%%      LABEL             blah blah blah
%%      <- width of ARG ->blah blah blah
%%                        blah blah blah
\newenvironment{describe}[1]%
   {\begin{list}{}{\settowidth{\labelwidth}{#1}%
            \setlength{\labelsep}{.5em}%
            \setlength{\leftmargin}{\labelwidth}% 
            \addtolength{\leftmargin}{\labelsep}%
            \addtolength{\leftmargin}{\parindent}%
            \def\makelabel##1{\rm ##1\hfill}}%
            \setlength{\topsep}{0pt}}%% 
                % Sets \topsep to 0 to reduce vertical space above
                % and below embedded displayed equations
   {\end{list}}

%   For tlatex.TeX
\usepackage{verbatim}
\makeatletter
\def\tla{\let\%\relax%
         \@bsphack
         \typeout{\%{\the\linewidth}}%
             \let\do\@makeother\dospecials\catcode`\^^M\active
             \let\verbatim@startline\relax
             \let\verbatim@addtoline\@gobble
             \let\verbatim@processline\relax
             \let\verbatim@finish\relax
             \verbatim@}
\let\endtla=\@esphack

\let\pcal=\tla
\let\endpcal=\endtla
\let\ppcal=\tla
\let\endppcal=\endtla

% The tlatex environment is used by TLATeX.TeX to typeset TLA+.
% TLATeX.TLA starts its files by writing a \tlatex command.  This
% command/environment sets \parindent to 0 and defines \% to its
% standard definition because the writing of the log files is messed up
% if \% is defined to be something else.  It also executes
% \@computerule to determine the dimensions for the TLA horizonatl
% bars.
\newenvironment{tlatex}{\@computerule%
                        \setlength{\parindent}{0pt}%
                       \makeatletter\chardef\%=`\%}{}


% The notla environment produces no output.  You can turn a 
% tla environment to a notla environment to prevent tlatex.TeX from
% re-formatting the environment.

\def\notla{\let\%\relax%
         \@bsphack
             \let\do\@makeother\dospecials\catcode`\^^M\active
             \let\verbatim@startline\relax
             \let\verbatim@addtoline\@gobble
             \let\verbatim@processline\relax
             \let\verbatim@finish\relax
             \verbatim@}
\let\endnotla=\@esphack

\let\nopcal=\notla
\let\endnopcal=\endnotla
\let\noppcal=\notla
\let\endnoppcal=\endnotla

%%%%%%%%%%%%%%%%%%%%%%%% end of tlatex.sty file %%%%%%%%%%%%%%%%%%%%%%% 
% last modified on Fri  3 August 2012 at 14:23:49 PST by lamport

\begin{document}
\tlatex
\setboolean{shading}{true}
\@x{}\moduleLeftDash\@xx{ {\MODULE} DistributedDB}\moduleRightDash\@xx{}%
\@pvspace{8.0pt}%
\@x{ {\EXTENDS} Utils}%
\@pvspace{8.0pt}%
\@x{ {\CONSTANTS} NumNodes ,\,\@s{16.81}}%
\@y{\@s{0}%
 number of nodes
}%
\@xx{}%
\@x{\@s{54.75} MaxQueueSize ,\,}%
\@y{\@s{0}%
 bound on size of message queues
}%
\@xx{}%
\@x{\@s{54.75} NumChains ,\,\@s{12.49}}%
\@y{\@s{0}%
 initial number of chain ids
}%
\@xx{}%
\@x{\@s{54.75} N\@s{61.48}}%
\@y{\@s{0}%
 bound on many things \ensuremath{\.{\stst}to} make model
 finite\ensuremath{\.{\stst}
}}%
\@xx{}%
\@pvspace{8.0pt}%
\@x{ {\VARIABLES}}%
\@y{\@s{0}%
 local information
}%
\@xx{}%
\@x{\@s{51.42} node\_info ,\,\@s{25.09}}%
\@y{\@s{0}%
 [ active : [ \ensuremath{Nodes \.{\rightarrow} {\SUBSET} Chains} ]
}%
\@xx{}%
\@x{\@s{125.31}}%
\@y{\@s{0}%
 , messages : [ \ensuremath{Nodes \.{\rightarrow}} [ \ensuremath{Chains
 \.{\rightarrow} Seq(Messages)} ] ]
}%
\@xx{}%
\@x{\@s{125.31}}%
\@y{\@s{0}%
 , branch : [ \ensuremath{Nodes \.{\rightarrow}} [ \ensuremath{Chains
 \.{\rightarrow} Branches} ] ]
}%
\@xx{}%
\@x{\@s{125.31}}%
\@y{\@s{0}%
 , blocks : [ \ensuremath{Nodes \.{\rightarrow}} [ \ensuremath{Chains
 \.{\rightarrow}} [ \ensuremath{Branches \.{\rightarrow} Seq(Blocks)} ] ] ] ]
}%
\@xx{}%
\@x{\@s{51.42}}%
\@y{\@s{0}%
 global view of network
}%
\@xx{}%
\@x{\@s{51.42} network\_info\@s{16.69}}%
\@y{\@s{0}%
 [ chains : \ensuremath{Chains
}}%
\@xx{}%
\@x{\@s{125.31}}%
\@y{\@s{0}%
 , branches : [ \ensuremath{Chains \.{\rightarrow} Branches} ]
}%
\@xx{}%
\@x{\@s{125.31}}%
\@y{\@s{0}%
 , heights : [ \ensuremath{Chains \.{\rightarrow}} [ \ensuremath{Branches
 \.{\rightarrow} Nat} ] ]
}%
\@xx{}%
\@x{\@s{125.31}}%
\@y{\@s{0}%
 , blocks : [ \ensuremath{Chains \.{\rightarrow}} [ \ensuremath{Branches
 \.{\rightarrow} Seq(Blocks)} ] ]
}%
\@xx{}%
\@x{\@s{125.31}}%
\@y{\@s{0}%
 , protocol : \ensuremath{Protocols
}}%
\@xx{}%
\@x{\@s{125.31}}%
\@y{\@s{0}%
 , active : [ \ensuremath{Chains \.{\rightarrow} {\SUBSET} Nodes} ]
}%
\@xx{}%
\@x{\@s{125.31}}%
\@y{\@s{0}%
 , sent\@s{12.5}: [ \ensuremath{Chains \.{\rightarrow}} [ \ensuremath{Nodes
 \.{\rightarrow} Seq(Messages)} ] ]
}%
\@xx{}%
\@x{\@s{125.31}}%
\@y{\@s{0}%
 , \ensuremath{recv}\@s{12.5}: [ \ensuremath{Chains \.{\rightarrow}} [
 \ensuremath{Nodes \.{\rightarrow} Seq(Messages)} ] ] ]
}%
\@xx{}%
\@pvspace{8.0pt}%
\@x{}%
\@y{\@s{0}%
 Associate a response to a request so we can determine if response is expected
}%
\@xx{}%
\@pvspace{8.0pt}%
\@x{}%
\@y{\@s{0}%
 Bootstrap pipeline with header/operation timeouts
 (\ensuremath{notify\_new\_block})
}%
\@xx{}%
\@pvspace{8.0pt}%
\@x{}%
\@y{%
 \ensuremath{{\ASSUME}} - \ensuremath{TODO
}}%
\@xx{}%
\@pvspace{8.0pt}%
\@x{ vars \.{\defeq} {\langle} network\_info ,\, node\_info {\rangle}}%
\@pvspace{8.0pt}%
\@x{ Nodes \.{\defeq} 1 \.{\dotdot} NumNodes}%
\@pvspace{8.0pt}%
\@x{ Chains \.{\defeq} 1 \.{\dotdot} NumChains}%
\@pvspace{8.0pt}%
\@x{ Branches \.{\defeq} 0 \.{\dotdot} N}%
\@pvspace{8.0pt}%
\@x{ Heights \.{\defeq} 0 \.{\dotdot} N}%
\@pvspace{8.0pt}%
\@x{ Protocols \.{\defeq} 0 \.{\dotdot} N}%
\@pvspace{8.0pt}%
\@x{ Headers \.{\defeq}}%
 \@x{\@s{16.4} [ height \.{:} Heights ,\, chain \.{:} Chains ,\, branch \.{:}
 Branches}%
 \@x{\@s{16.4} ,\, protocol \.{:} Protocols ,\, num\_ops \.{:} 0 \.{\dotdot} N
 ]}%
\@pvspace{8.0pt}%
 \@x{ Operations \.{\defeq} [ 0 \.{\dotdot} N \.{\rightarrow} Pairs ( Heights
 ,\, 0 \.{\dotdot} N ) ]}%
\@pvspace{8.0pt}%
\@x{ Blocks \.{\defeq} [ header \.{:} Headers ,\, ops \.{:} Operations ]}%
\@pvspace{8.0pt}%
\@x{}\midbar\@xx{}%
\@pvspace{8.0pt}%
\@x{}%
\@y{\@s{0}%
 Module-specific helper functions
}%
\@xx{}%
\@x{}%
\@y{\@s{0}%
 check that [node]\mbox{'}s message queue on [chain] is not full
}%
\@xx{}%
\@x{ checkNodeQueue [ node \.{\in} Nodes ] \.{\defeq}}%
 \@x{\@s{16.4} [ chain \.{\in} Chains \.{\mapsto} Len ( node\_info . messages
 [ chain ] [ node ] ) \.{<} MaxQueueSize ]}%
\@pvspace{8.0pt}%
\@x{}%
\@y{\@s{0}%
 check that there is space to send [node] a message on [chain]
}%
\@xx{}%
\@x{ checkNetQueue [ chain \.{\in} Chains ] \.{\defeq}}%
 \@x{\@s{16.4} [ node \.{\in} Nodes \.{\mapsto} Len ( network\_info . sent [
 chain ] [ node ] ) \.{<} MaxQueueSize ]}%
\@pvspace{8.0pt}%
\@x{ {\RECURSIVE} \_update\_msgs ( \_ ,\, \_ ,\, \_ )}%
\@x{ \_update\_msgs ( msgs ,\, msg ,\, to\_update ) \.{\defeq}}%
\@x{\@s{16.4} {\CASE} to\_update \.{=} \{ \} \.{\rightarrow} msgs}%
\@x{\@s{16.4} {\Box} {\OTHER} \.{\rightarrow}}%
\@x{\@s{32.07} \.{\LET}\@s{5.17} p \.{\defeq} Pick ( to\_update )}%
 \@x{\@s{57.64} new \.{\defeq} [ msgs {\EXCEPT} {\bang} [ p ] \.{=} Append ( @
 ,\, msg ) ]}%
 \@x{\@s{32.07} \.{\IN}\@s{5.17} \_update\_msgs ( new ,\, msg ,\, to\_update
 \.{\,\backslash\,} \{ p \} )}%
\@pvspace{8.0pt}%
\@x{ update\_msgs ( msgs ,\, chain ,\, msg ) \.{\defeq}}%
\@x{\@s{16.4} \.{\LET} from \.{\defeq} msg . from}%
 \@x{\@s{36.79} active \.{\defeq} network\_info . active [ chain ]
 \.{\,\backslash\,} \{ from \}}%
\@x{\@s{40.89} receivers \.{\defeq}}%
 \@x{\@s{49.09} \{ node \.{\in} active \.{:} checkNetQueue [ chain ] [ node ]
 \}}%
\@x{\@s{16.4} \.{\IN}\@s{4.09} \_update\_msgs ( msgs ,\, msg ,\, receivers )}%
\@pvspace{8.0pt}%
\@x{}%
\@y{\@s{0}%
 check that queue is not full before appending the message
}%
\@xx{}%
\@x{ checkAppend ( queue ,\, msg ) \.{\defeq}}%
 \@x{\@s{16.4} {\CASE} Len ( queue ) \.{<} MaxQueueSize \.{\rightarrow} Append
 ( queue ,\, msg )}%
\@x{\@s{24.59} {\Box}\@s{10.30} {\OTHER} \.{\rightarrow} queue}%
\@pvspace{8.0pt}%
\@x{}%
\@y{\@s{0}%
 Blocks
}%
\@xx{}%
 \@x{ Header ( \_height ,\, \_chain ,\, \_branch ,\, \_protocol ,\, \_num\_ops
 ) \.{\defeq}}%
 \@x{\@s{16.4} [ height \.{\mapsto} \_height ,\, chain \.{\mapsto} \_chain ,\,
 branch \.{\mapsto} \_branch}%
 \@x{\@s{16.4} ,\, protocol \.{\mapsto} \_protocol ,\, num\_ops \.{\mapsto}
 \_num\_ops ]}%
\@pvspace{8.0pt}%
 \@x{ Block ( \_header ,\, \_ops ) \.{\defeq} [ header \.{\mapsto} \_header
 ,\, ops \.{\mapsto} \_ops ]}%
\@pvspace{8.0pt}%
 \@x{ mkOps ( \_height ,\, \_num\_ops ) \.{\defeq} [ x \.{\in} 1 \.{\dotdot}
 \_num\_ops \.{\mapsto} {\langle} \_height ,\, x {\rangle} ]}%
\@pvspace{8.0pt}%
\@x{ isNetwork ( info ) \.{\defeq} {\DOMAIN} info \.{=}}%
 \@x{\@s{16.4} \{\@w{active} ,\,\@w{branches} ,\,\@w{blocks} ,\,\@w{chains}
 ,\,\@w{protocol} ,\,\@w{recv} ,\,\@w{sent} \}}%
\@pvspace{8.0pt}%
 \@x{ isNode ( info ) \.{\defeq} {\DOMAIN} info \.{=} \{\@w{active}
 ,\,\@w{branch} ,\,\@w{blocks} ,\,\@w{messages} \}}%
\@pvspace{8.0pt}%
\@x{}%
\@y{\@s{0}%
 If \ensuremath{info \.{=} node\_info}, check that
 \ensuremath{checkNodeQueue[node][chain]} is satisified
}%
\@xx{}%
\@x{}%
\@y{\@s{0}%
 If \ensuremath{info \.{=} network\_info}, check that
 \ensuremath{checkNetQueue[chain][node]} is satisified
}%
\@xx{}%
\@x{ Recv ( info ,\, node ,\, chain ,\, msg ) \.{\defeq}}%
\@x{\@s{21.14}}%
\@y{\@s{0}%
 \ensuremath{node\_info
}}%
\@xx{}%
\@x{\@s{21.14} {\CASE} isNode ( info )}%
\@x{\@s{37.54} \.{\rightarrow} [ info {\EXCEPT} {\bang} . messages \.{=}}%
\@x{\@s{64.08} [ @ {\EXCEPT} {\bang} [ node ] \.{=}}%
 \@x{\@s{74.63} [ @ {\EXCEPT} {\bang} [ chain ] \.{=} Append ( @ ,\, msg ) ] ]
 ]}%
\@x{\@s{21.14}}%
\@y{\@s{0}%
 \ensuremath{network\_info
}}%
\@xx{}%
\@x{\@s{29.34} {\Box} isNetwork ( info )}%
\@x{\@s{37.54} \.{\rightarrow} [ info {\EXCEPT} {\bang} . recv \.{=}}%
\@x{\@s{64.08} [ @ {\EXCEPT} {\bang} [ chain ] \.{=}}%
 \@x{\@s{74.63} [ @ {\EXCEPT} {\bang} [ node ] \.{=} {\langle} msg {\rangle}
 \.{\circ} @ ] ] ]}%
\@pvspace{8.0pt}%
\@x{}%
\@y{\@s{0}%
 If \ensuremath{info \.{=} network\_info}, must check that
 \ensuremath{network\_info.sent[chain][node] \.{\neq} {\langle}{\rangle}
}}%
\@xx{}%
\@x{}%
\@y{\@s{0}%
 If \ensuremath{info \.{=} node\_info}, must check that
 \ensuremath{node\_info.sent[node][chain] \.{\neq} {\langle}{\rangle}
}}%
\@xx{}%
\@x{ Consume ( info ,\, node ,\, chain ) \.{\defeq}}%
\@x{\@s{16.4}}%
\@y{\@s{0}%
 \ensuremath{network\_info
}}%
\@xx{}%
\@x{\@s{16.4} {\CASE} isNetwork ( info )}%
\@x{\@s{32.8} \.{\rightarrow} [ info {\EXCEPT} {\bang} . sent \.{=}}%
\@x{\@s{59.33} [ @ {\EXCEPT} {\bang} [ chain ] \.{=}}%
\@x{\@s{69.88} [ @ {\EXCEPT} {\bang} [ node ] \.{=} Tail ( @ ) ] ] ]}%
\@x{\@s{16.4}}%
\@y{\@s{0}%
 \ensuremath{node\_info
}}%
\@xx{}%
\@x{\@s{24.59} {\Box} isNode ( info ) \.{\rightarrow}}%
\@x{\@s{40.27} [ info {\EXCEPT} {\bang} . messages \.{=}}%
\@x{\@s{51.25} [ @ {\EXCEPT} {\bang} [ node ] \.{=}}%
\@x{\@s{61.80} [ @ {\EXCEPT} {\bang} [ chain ] \.{=} Tail ( @ ) ] ] ]}%
\@pvspace{8.0pt}%
\@x{}%
\@y{\@s{0}%
 Must check that \ensuremath{checkNetQueue[chain][node]} is satisfied
}%
\@xx{}%
\@x{ Sent ( info ,\, node ,\, chain ,\, msg ) \.{\defeq}}%
\@x{\@s{20.11} {\CASE} isNetwork ( info )}%
\@x{\@s{36.51} \.{\rightarrow} [ info {\EXCEPT} {\bang} . sent \.{=}}%
\@x{\@s{63.04} [ @ {\EXCEPT} {\bang} [ chain ] \.{=}}%
 \@x{\@s{73.60} [ @ {\EXCEPT} {\bang} [ node ] \.{=} Append ( @ ,\, msg ) ] ]
 ]}%
\@pvspace{8.0pt}%
\@x{}%
\@y{}%
\@xx{}%
\@x{ CheckSent ( info ,\, node ,\, chain ,\, msg ) \.{\defeq}}%
\@x{\@s{16.4} {\CASE}\@s{3.29} isNetwork ( info )}%
\@x{\@s{32.8} \.{\rightarrow} [ info {\EXCEPT} {\bang} . sent \.{=}}%
\@x{\@s{59.33} [ @ {\EXCEPT} {\bang} [ chain ] \.{=}}%
\@x{\@s{69.88} [ to \.{\in} Nodes \.{\mapsto}}%
\@x{\@s{80.86} \.{\LET} curr \.{\defeq} @ [ to ]}%
\@x{\@s{80.86} \.{\IN}}%
\@y{\@s{0}%
 if active and has queue space, append \ensuremath{msg
}}%
\@xx{}%
 \@x{\@s{89.06} {\CASE} to \.{\in} info . active [ chain ] \.{\,\backslash\,}
 \{ node \}}%
\@x{\@s{105.46} \.{\rightarrow} checkAppend ( curr ,\, msg )}%
\@x{\@s{93.16}}%
\@y{\@s{0}%
 otherwise, do nothing
}%
\@xx{}%
\@x{\@s{97.26} {\Box} {\OTHER} \.{\rightarrow} curr ] ] ]}%
\@pvspace{8.0pt}%
\@x{}\midbar\@xx{}%
\@pvspace{8.0pt}%
\@x{}%
\@y{\@s{0}%
 Messages
}%
\@xx{}%
\@pvspace{8.0pt}%
\@x{}%
\@y{\@s{0}%
 Advertise messages
}%
\@xx{}%
\begin{lcom}{2.5}%
\begin{cpar}{0}{F}{F}{0}{0}{}%
 - \ensuremath{Current\_branch} of \ensuremath{Chain\_id.t} *
 \ensuremath{Block\_locator.t
}%
\end{cpar}%
\begin{cpar}{0}{F}{F}{0}{0}{}%
 - \ensuremath{Current\_head} of \ensuremath{Chain\_id.t} *
 \ensuremath{Block\_header.t} * \ensuremath{Mempool.t
}%
\end{cpar}%
\begin{cpar}{0}{F}{F}{0}{0}{}%
- \ensuremath{Block\_header} of \ensuremath{Block\_header.t
}%
\end{cpar}%
\begin{cpar}{0}{F}{F}{0}{0}{}%
- \ensuremath{Operation} of \ensuremath{Operation.t
}%
\end{cpar}%
\begin{cpar}{0}{F}{F}{0}{0}{}%
- \ensuremath{Protocol} of \ensuremath{Protocol.t
}%
\end{cpar}%
\begin{cpar}{0}{T}{F}{0}{12.5}{}%
- \ensuremath{Operations\_for\_block} of
 \ensuremath{Block\_hash.t} * int * \ensuremath{Operation.t} list *
 \ensuremath{Operation\_list\_list\_hash.path
}%
\end{cpar}%
\begin{cpar}{1}{F}{F}{0}{0}{}%
 - \ensuremath{Checkpoint} of \ensuremath{Chain\_id.t} *
 \ensuremath{Block\_header.t
}%
\end{cpar}%
\begin{cpar}{0}{T}{F}{0}{12.5}{}%
- \ensuremath{Protocol\_branch} of
 \ensuremath{Chain\_id.t} * int * \ensuremath{Block\_locator.t
}%
\end{cpar}%
\begin{cpar}{1}{F}{F}{0}{0}{}%
 - \ensuremath{Predecessor\_header} of \ensuremath{Block\_hash.t} *
 \ensuremath{int32} * \ensuremath{Block\_header.t
}%
\end{cpar}%
\end{lcom}%
\@x{ Locators\@s{8.71} \.{\defeq} \{\@w{locator} \}\@s{34.37}}%
\@y{\@s{0}%
 what is this\.{?}
}%
\@xx{}%
\@pvspace{8.0pt}%
\@x{ Mempools\@s{2.31} \.{\defeq} \{\@w{mempool} \}\@s{24.60}}%
\@y{\@s{0}%
 is this associated with a block or node\.{?}
}%
\@xx{}%
\@pvspace{8.0pt}%
\@x{ AdParams \.{\defeq}}%
\@x{\@s{16.4} [ chain \.{:} Chains ,\,\@s{4.1} locator \.{:} Locators ]}%
\@x{\@s{16.4} \.{\cup}}%
 \@x{\@s{16.4} [ chain \.{:} Chains ,\, header \.{:} Headers ,\, mempool \.{:}
 Mempools ]}%
\@x{\@s{16.4} \.{\cup}}%
\@x{\@s{16.4} [ header \.{:} Headers ]}%
\@x{\@s{16.4} \.{\cup}}%
\@x{\@s{16.4} [ operation \.{:} Operations ]}%
\@x{\@s{16.4} \.{\cup}}%
\@x{\@s{16.4} [ protocol \.{:} Protocols ]}%
\@x{\@s{16.4} \.{\cup}}%
\@x{\@s{16.4} [ height \.{:} Heights ,\,}%
\@y{%
 branch\.{?},
}%
\@xx{ ops \.{:} NESeq\_n ( Operations ,\, N )}%
\@x{\@s{16.4} ,\, ops\_hash \.{:} NESeq\_n ( Heights ,\, N ) ]}%
\@x{\@s{16.4} \.{\cup}}%
\@x{\@s{16.4} [ chain \.{:} Chains ,\, header \.{:} Headers ]}%
\@x{\@s{16.4} \.{\cup}}%
\@x{\@s{16.4} [ chain \.{:} Chains ,\,}%
\@y{%
 int\.{?},
}%
\@xx{ locator \.{:} Locators ]}%
\@x{\@s{16.4} \.{\cup}}%
\@x{\@s{16.4} [ height \.{:} Heights ,\,}%
\@y{%
 \ensuremath{int32}\.{?},
}%
\@xx{ header \.{:} Headers ]}%
\@pvspace{8.0pt}%
\@x{ AdMsgTypes \.{\defeq}}%
 \@x{\@s{16.4} \{\@w{Current\_branch} ,\,\@w{Current\_head}
 ,\,\@w{Block\_header} ,\,\@w{Operation}}%
 \@x{\@s{16.4} ,\,\@w{Protocol} ,\,\@w{Operations\_for\_block}
 ,\,\@w{Checkpoint} ,\,\@w{Protocol\_branch}}%
\@x{\@s{16.4} ,\,\@w{Predecessor\_header} \}}%
\@pvspace{8.0pt}%
 \@x{ AdMsgs \.{\defeq} [ from \.{:} Nodes ,\, type \.{:} AdMsgTypes ,\,
 params \.{:} AdParams ]}%
\@pvspace{8.0pt}%
\@x{}%
\@y{\@s{0}%
 Request messages
}%
\@xx{}%
\begin{lcom}{2.5}%
\begin{cpar}{0}{F}{F}{0}{0}{}%
- \ensuremath{Get\_current\_branch} of \ensuremath{Chain\_id.t
}%
\end{cpar}%
\begin{cpar}{0}{F}{F}{0}{0}{}%
- \ensuremath{Get\_current\_head} of \ensuremath{Chain\_id.t
}%
\end{cpar}%
\begin{cpar}{0}{F}{F}{0}{0}{}%
- \ensuremath{Get\_checkpoint} of \ensuremath{Chain\_id.t
}%
\end{cpar}%
\begin{cpar}{0}{F}{F}{0}{0}{}%
- \ensuremath{Get\_protocol\_branch} of \ensuremath{Chain\_id.t} * int
\end{cpar}%
\begin{cpar}{0}{F}{F}{0}{0}{}%
- \ensuremath{Get\_block\_headers} of \ensuremath{Block\_hash.t} list
\end{cpar}%
\begin{cpar}{0}{F}{F}{0}{0}{}%
- \ensuremath{Get\_operations} of \ensuremath{Operation\_hash.t} list
\end{cpar}%
\begin{cpar}{0}{F}{F}{0}{0}{}%
- \ensuremath{Get\_protocols} of \ensuremath{Protocol\_hash.t} list
\end{cpar}%
\begin{cpar}{0}{F}{F}{0}{0}{}%
 - \ensuremath{Get\_operations\_for\_blocks} of (\ensuremath{Block\_hash.t} *
 int) list
\end{cpar}%
\begin{cpar}{0}{F}{F}{0}{0}{}%
 - \ensuremath{Get\_predecessor\_header} of \ensuremath{Block\_hash.t} *
 \ensuremath{int32
}%
\end{cpar}%
\end{lcom}%
\@x{ ReqParams \.{\defeq}}%
\@x{\@s{16.4} [ chain \.{:} Chains ]\@s{175.55}}%
\@y{\@s{0}%
 \ensuremath{current\_branch}, \ensuremath{current\_head}, checkpoint
}%
\@xx{}%
\@x{\@s{16.4} \.{\cup}}%
 \@x{\@s{16.4} [ chain \.{:} Chains ,\, branch \.{:} 1 \.{\dotdot} N
 ]\@s{104.74}}%
\@y{\@s{0}%
 branch\.{?} for \ensuremath{Get\_protocol\_branch
}}%
\@xx{}%
\@x{\@s{16.4} \.{\cup}}%
\@x{\@s{16.4} [ heights \.{:} NESeq\_n ( Heights ,\, N ) ]\@s{103.78}}%
\@y{\@s{0}%
 headers, operations, protocols
}%
\@xx{}%
\@x{\@s{16.4} \.{\cup}}%
 \@x{\@s{16.4} [ height\_num\_list \.{:} NESeq\_n ( Pairs ( Heights ,\, 1
 \.{\dotdot} N ) ,\, N ) ]}%
\@y{\@s{0}%
 \ensuremath{num}\.{?}
}%
\@xx{}%
\@x{\@s{16.4} \.{\cup}}%
 \@x{\@s{16.4} [ height \.{:} Heights ,\, num \.{:} 1 \.{\dotdot} N
 ]\@s{109.77}}%
\@y{\@s{0}%
 \ensuremath{num}\.{?}
}%
\@xx{}%
\@pvspace{8.0pt}%
 \@x{ OnlyChain \.{\defeq} \{\@w{Get\_current\_branch}
 ,\,\@w{Get\_current\_head} ,\,\@w{Get\_checkpoint} \}}%
\@pvspace{8.0pt}%
 \@x{ OnlyHeights \.{\defeq} \{\@w{Get\_block\_headers}
 ,\,\@w{Get\_operations} ,\,\@w{Get\_protocols} \}}%
\@pvspace{8.0pt}%
\@x{ ChainAndBranch \.{\defeq} \{\@w{Get\_protocol\_branch} \}}%
\@pvspace{8.0pt}%
\@x{ HeightNumList \.{\defeq} \{\@w{Get\_operations\_for\_blocks} \}}%
\@pvspace{8.0pt}%
\@x{ HeightAndNum \.{\defeq} \{\@w{Get\_predecessor\_header} \}}%
\@pvspace{8.0pt}%
\@x{ ReqMsgTypes \.{\defeq}}%
 \@x{\@s{16.4} OnlyChain \.{\cup} OnlyHeights \.{\cup} ChainAndBranch \.{\cup}
 HeightNumList \.{\cup} HeightAndNum}%
\@pvspace{8.0pt}%
 \@x{ ReqMsgs \.{\defeq} [ from \.{:} Nodes ,\, type \.{:} ReqMsgTypes ,\,
 params \.{:} ReqParams ]}%
\@pvspace{8.0pt}%
\@x{}%
\@y{\@s{0}%
 All messages
}%
\@xx{}%
\@x{ Messages \.{\defeq} AdMsgs \.{\cup} ReqMsgs}%
\@pvspace{8.0pt}%
\@x{}%
\@y{\@s{0}%
 Queues can have size at most \ensuremath{MaxQueueSize
}}%
\@xx{}%
\@x{ PossibleQueueStates \.{\defeq} Seq\_n ( Messages ,\, MaxQueueSize )}%
\@pvspace{8.0pt}%
\@x{}%
\@y{\@s{0}%
 validates \ensuremath{\_type} matches \ensuremath{\_params} and creates the
 message
}%
\@xx{}%
\@x{}%
\@y{\@s{0}%
 invalid type/param pairs will return a \ensuremath{TLC} error
}%
\@xx{}%
\@x{ Msg ( \_from ,\, \_type ,\, \_params ) \.{\defeq}}%
\@x{\@s{22.42} {\CASE} \.{\lor} \.{\land} \_type \.{\in} OnlyChain}%
\@x{\@s{59.51} \.{\land} {\DOMAIN} \_params \.{=} \{\@w{chain} \}}%
\@x{\@s{48.40} \.{\lor} \.{\land} \_type \.{\in} OnlyHeights}%
\@x{\@s{59.51} \.{\land} {\DOMAIN} \_params \.{=} \{\@w{heights} \}}%
\@x{\@s{48.40} \.{\lor} \.{\land} \_type \.{\in} ChainAndBranch}%
 \@x{\@s{59.51} \.{\land} {\DOMAIN} \_params \.{=} \{\@w{chain} ,\,\@w{branch}
 \}}%
\@x{\@s{48.40} \.{\lor} \.{\land} \_type \.{\in} HeightNumList}%
 \@x{\@s{59.51} \.{\land} {\DOMAIN} \_params \.{=} \{\@w{height\_num\_list}
 \}}%
\@x{\@s{48.40} \.{\lor} \.{\land} \_type \.{\in} HeightAndNum}%
 \@x{\@s{63.96} \.{\land} {\DOMAIN} \_params \.{=} \{\@w{height} ,\,\@w{num}
 \}}%
 \@x{\@s{48.40} \.{\rightarrow} [ from \.{\mapsto} \_from ,\, type \.{\mapsto}
 \_type ,\, params \.{\mapsto} \_params ]}%
\@pvspace{8.0pt}%
\@x{}\midbar\@xx{}%
\@pvspace{8.0pt}%
\@x{}%
\@y{\@s{0}%
 Activate/\ensuremath{Deactivate} messages are not explicitly passed between
 nodes in this model
}%
\@xx{}%
\@pvspace{8.0pt}%
\@x{}%
\@y{\@s{0}%
 An inactive node becomes active on given chain
}%
\@xx{}%
\@x{ Activate ( node ,\, chain ) \.{\defeq}}%
\@x{\@s{16.4} \.{\land} network\_info \.{'} \.{=}}%
 \@x{\@s{35.71} [ network\_info {\EXCEPT} {\bang} . active \.{=} [ @ {\EXCEPT}
 {\bang} [ chain ] \.{=} @ \.{\cup} \{ node \} ] ]}%
\@x{\@s{16.4} \.{\land} node\_info \.{'} \.{=}}%
 \@x{\@s{35.71} [ node\_info {\EXCEPT} {\bang} . active \.{=} [ @ {\EXCEPT}
 {\bang} [ node ] \.{=} @ \.{\cup} \{ chain \} ] ]}%
\@pvspace{8.0pt}%
\@x{}%
\@y{\@s{0}%
 An active node becomes inactive on given chain
}%
\@xx{}%
\@x{ Deactivate ( node ,\, chain ) \.{\defeq}}%
\@x{\@s{16.4} \.{\land} network\_info \.{'} \.{=}}%
 \@x{\@s{35.71} [ network\_info {\EXCEPT} {\bang} . active \.{=} [ @ {\EXCEPT}
 {\bang} [ chain ] \.{=} @ \.{\,\backslash\,} \{ node \} ] ]}%
\@x{\@s{16.4} \.{\land} node\_info \.{'} \.{=}}%
 \@x{\@s{35.71} [ node\_info {\EXCEPT} {\bang} . active \.{=} [ @ {\EXCEPT}
 {\bang} [ node ] \.{=} @ \.{\,\backslash\,} \{ chain \} ] ]}%
\@pvspace{8.0pt}%
\@x{}\midbar\@xx{}%
\@pvspace{8.0pt}%
\@x{}%
\@y{\@s{0}%
 Request actions
}%
\@xx{}%
\@x{}%
\@y{\@s{0}%
 Messages are sent to a queue where the receipient can receive the message
}%
\@xx{}%
\@pvspace{8.0pt}%
\@x{}%
\@y{\@s{0}%
 Request the current branch from one active peer
}%
\@xx{}%
\@x{}%
\@y{\@s{0}%
 must check that [to] has room left in their queue
}%
\@xx{}%
\@x{ Get\_current\_branch\_1 ( from ,\, chain ,\, to ) \.{\defeq}}%
 \@x{\@s{16.4} \.{\LET} msg \.{\defeq} Msg ( from ,\,\@w{Get\_current\_branch}
 ,\, [ chain \.{\mapsto} chain ] )}%
\@x{\@s{16.4} \.{\IN}}%
 \@x{\@s{24.59} \.{\land} network\_info \.{'} \.{=} Sent ( network\_info ,\,
 to ,\, chain ,\, msg )}%
\@x{\@s{24.59} \.{\land} {\UNCHANGED} node\_info}%
\@pvspace{8.0pt}%
\@x{}%
\@y{\@s{0}%
 Request the current branch from all active peers
}%
\@xx{}%
\@x{ Get\_current\_branch\_n ( from ,\, chain ) \.{\defeq}}%
 \@x{\@s{16.4} \.{\LET} msg \.{\defeq} Msg ( from ,\,\@w{Get\_current\_branch}
 ,\, [ chain \.{\mapsto} chain ] )}%
\@x{\@s{16.4} \.{\IN}}%
 \@x{\@s{24.59} \.{\land} network\_info \.{'} \.{=} CheckSent ( network\_info
 ,\, from ,\, chain ,\, msg )}%
\@x{\@s{24.59} \.{\land} {\UNCHANGED} node\_info}%
\@pvspace{8.0pt}%
\@x{}%
\@y{\@s{0}%
 Request the current head from one active peer
}%
\@xx{}%
\@x{ Get\_current\_head\_1 ( from ,\, chain ,\, to ) \.{\defeq}}%
 \@x{\@s{16.4} \.{\LET} msg \.{\defeq} Msg ( from ,\,\@w{Get\_current\_head}
 ,\, [ chain \.{\mapsto} chain ] )}%
\@x{\@s{16.4} \.{\IN}}%
 \@x{\@s{24.59} \.{\land} network\_info \.{'} \.{=} Sent ( network\_info ,\,
 to ,\, chain ,\, msg )}%
\@x{\@s{24.59} \.{\land} {\UNCHANGED} node\_info}%
\@pvspace{8.0pt}%
\@x{}%
\@y{\@s{0}%
 Request the current head from all active peers
}%
\@xx{}%
\@x{ Get\_current\_head\_n ( from ,\, chain ) \.{\defeq}}%
 \@x{\@s{16.4} \.{\LET} msg \.{\defeq} Msg ( from ,\,\@w{Get\_current\_head}
 ,\, [ chain \.{\mapsto} chain ] )}%
\@x{\@s{16.4} \.{\IN}}%
 \@x{\@s{24.59} \.{\land} network\_info \.{'} \.{=} CheckSent ( network\_info
 ,\, from ,\, chain ,\, msg )}%
\@x{\@s{24.59} \.{\land} {\UNCHANGED} node\_info}%
\@pvspace{8.0pt}%
\@x{}\midbar\@xx{}%
\@pvspace{8.0pt}%
\@x{}%
\@y{\@s{0}%
 Advertise actions
}%
\@xx{}%
\@x{}%
\@y{\@s{0}%
 Advertise messages are explicitly passed between nodes
}%
\@xx{}%
\@pvspace{8.0pt}%
\@x{}%
\@y{\@s{0}%
 \ensuremath{TODO
}}%
\@xx{}%
\@x{}%
\@y{\@s{0}%
 These messages do not only serve as responses to requests
}%
\@xx{}%
\@x{}%
\@y{\@s{0}%
 They are also broadcast to the active nodes of the chain\.{?}
}%
\@xx{}%
\@pvspace{8.0pt}%
\@x{}\midbar\@xx{}%
\@pvspace{8.0pt}%
\@x{}%
\@y{\@s{0}%
 Receiving messages
}%
\@xx{}%
\@x{ Receive\_msg ( node ,\, chain ) \.{\defeq}}%
 \@x{\@s{16.4} \.{\LET} recv\_q \.{\defeq} network\_info . sent [ chain ] [
 node ]}%
\@x{\@s{36.79} in\_q \.{\defeq} node\_info . messages [ node ] [ chain ]}%
\@x{\@s{16.4} \.{\IN}}%
\@x{\@s{24.59} \.{\land} recv\_q \.{\neq} {\langle} {\rangle}\@s{92.30}}%
\@y{\@s{0}%
 there are messages for [node] to receive
}%
\@xx{}%
\@x{\@s{24.59} \.{\land} checkNodeQueue [ node ] [ chain ]\@s{12.29}}%
\@y{\@s{0}%
 [node] has space for incoming messages
}%
\@xx{}%
\@x{\@s{24.59} \.{\land} \.{\LET} msg \.{\defeq} Head ( recv\_q )}%
\@x{\@s{35.71} \.{\IN}}%
\@x{\@s{43.91}}%
\@y{\@s{0}%
 [\ensuremath{msg}] is added to the end of [node]\mbox{'}s queue
}%
\@xx{}%
\@x{\@s{43.91} \.{\land} node\_info \.{'} \.{=}}%
\@x{\@s{63.22} [ node\_info {\EXCEPT} {\bang} . messages \.{=}}%
 \@x{\@s{74.19} [ @ {\EXCEPT} {\bang} [ node ] \.{=} [ @ {\EXCEPT} {\bang} [
 chain ] \.{=} Append ( @ ,\, msg ) ] ] ]}%
\@x{\@s{43.91}}%
\@y{\@s{0}%
 [\ensuremath{msg}] is removed from [node]\mbox{'}s sent and add to the
 beginning of \ensuremath{recv
}}%
\@xx{}%
 \@x{\@s{43.91} \.{\land} \.{\LET} new \.{\defeq} Consume ( network\_info ,\,
 node ,\, chain )}%
\@x{\@s{75.42} new\_info \.{\defeq} Recv ( new ,\, node ,\, chain ,\, msg )}%
\@x{\@s{55.02} \.{\IN} network\_info \.{'} \.{=} new\_info}%
\@pvspace{8.0pt}%
\@x{}\midbar\@xx{}%
\@pvspace{8.0pt}%
\@x{}%
\@y{\@s{0}%
 Protocol upgrade
}%
\@xx{}%
\@x{ Protocol\_upgrade \.{\defeq}}%
\@x{\@s{16.4} \.{\land} network\_info . protocol \.{<} N}%
 \@x{\@s{16.4} \.{\land} network\_info \.{'} \.{=} [ network\_info {\EXCEPT}
 {\bang} . protocol \.{=} @ \.{+} 1 ]}%
\@x{\@s{16.4} \.{\land} {\UNCHANGED} node\_info}%
\@pvspace{8.0pt}%
\@x{}%
\@y{\@s{0}%
 Block production
}%
\@xx{}%
\@x{ Produce\_block ( chain ,\, branch ,\, num\_ops ) \.{\defeq}}%
\@x{\@s{16.4} \.{\LET} proto \.{\defeq} network\_info . protocol\@s{81.97}}%
\@y{\@s{0}%
 current protocol
}%
\@xx{}%
 \@x{\@s{36.79} height \.{\defeq} network\_info . heights [ chain ] [ branch ]
 \.{+} 1}%
\@y{\@s{0}%
 next block height
}%
\@xx{}%
\@x{\@s{36.79} ops \.{\defeq} mkOps ( height ,\, num\_ops )}%
 \@x{\@s{36.79} header \.{\defeq} Header ( height ,\, chain ,\, branch ,\,
 proto ,\, num\_ops )}%
\@x{\@s{16.4} \.{\IN}}%
\@x{\@s{24.59} \.{\land} height \.{\leq} N}%
\@x{\@s{24.59} \.{\land} network\_info \.{'} \.{=}}%
\@x{\@s{43.91} [ network\_info {\EXCEPT}}%
 \@x{\@s{54.88} {\bang} . blocks\@s{5.34} \.{=} [ c \.{\in} Chains \.{\mapsto}
 [ b \.{\in} network\_info . branches [ c ] \.{\mapsto}}%
 \@x{\@s{62.23} {\CASE} b \.{=} branch \.{\rightarrow} Append ( @ ,\, Block (
 header ,\, ops ) )}%
\@x{\@s{70.43} {\Box}\@s{10.30} {\OTHER} \.{\rightarrow} @ ] ] ]}%
\@x{\@s{24.59} \.{\land} {\UNCHANGED} node\_info}%
\@pvspace{8.0pt}%
\@x{}%
\@y{\@s{0}%
 New branch
}%
\@xx{}%
\@x{ New\_branch\_for ( chain ) \.{\defeq}}%
\@x{\@s{16.4} \.{\land} network\_info . branches [ chain ] \.{<} N}%
\@x{\@s{16.4} \.{\land} network\_info \.{'} \.{=}}%
 \@x{\@s{35.71} [ network\_info {\EXCEPT} {\bang} . branches \.{=} [ @
 {\EXCEPT} {\bang} [ chain ] \.{=} @ \.{+} 1 ] ]}%
\@x{\@s{16.4} \.{\land} {\UNCHANGED} node\_info}%
\@pvspace{8.0pt}%
\@x{}\midbar\@xx{}%
\@pvspace{8.0pt}%
\@x{}%
\@y{\@s{0}%
 \ensuremath{TODO
}}%
\@xx{}%
\@x{}%
\@y{\@s{0}%
 A node that is inactive on a chain becomes active
}%
\@xx{}%
\@x{ Activation \.{\defeq}}%
\@x{\@s{16.4} \E\, chain \.{\in} Chains \.{:}}%
 \@x{\@s{27.72} \E\, node \.{\in} Nodes \.{\,\backslash\,} network\_info .
 active [ chain ] \.{:} Activate ( node ,\, chain )}%
\@pvspace{8.0pt}%
\@x{}%
\@y{\@s{0}%
 A node that is active on a chain becomes inactive
}%
\@xx{}%
\@x{ Deactivation \.{\defeq}}%
\@x{\@s{16.4} \E\, chain \.{\in} Chains \.{:}}%
 \@x{\@s{27.72} \E\, node \.{\in} network\_info . active [ chain ] \.{:}
 Deactivate ( node ,\, chain )}%
\@pvspace{8.0pt}%
\@x{}%
\@y{\@s{0}%
 Request current branch from an active peer who can receive a message
}%
\@xx{}%
\@x{ Get\_current\_branch\_one \.{\defeq}}%
\@x{\@s{16.4} \E\, from \.{\in} Nodes ,\, chain \.{\in} Chains \.{:}}%
 \@x{\@s{27.72} \E\, to \.{\in} network\_info . active [ chain ]
 \.{\,\backslash\,} \{ from \} \.{:}}%
\@x{\@s{39.04} \.{\land} checkNetQueue [ chain ] [ to ]}%
\@x{\@s{39.04} \.{\land} Get\_current\_branch\_1 ( from ,\, chain ,\, to )}%
\@pvspace{8.0pt}%
\@x{}%
\@y{\@s{0}%
 Request current branch from all active peers
}%
\@xx{}%
\@x{ Get\_current\_branch\_all \.{\defeq}}%
\@x{\@s{16.4} \E\, from \.{\in} Nodes \.{:}}%
\@x{\@s{27.72} \E\, chain \.{\in} node\_info . active [ from ] \.{:}}%
 \@x{\@s{39.04} \.{\land} network\_info . active [ chain ] \.{\,\backslash\,}
 \{ from \} \.{\neq} \{ \}}%
\@x{\@s{39.04} \.{\land} Get\_current\_branch\_n ( from ,\, chain )}%
\@pvspace{8.0pt}%
\@x{}%
\@y{\@s{0}%
 Request current head from an active peer
}%
\@xx{}%
\@x{ Get\_current\_head\_one \.{\defeq}}%
\@x{\@s{16.4} \E\, from \.{\in} Nodes \.{:}}%
\@x{\@s{27.72} \E\, chain \.{\in} node\_info . active [ from ] \.{:}}%
 \@x{\@s{39.04} \E\, to \.{\in} network\_info . active [ chain ]
 \.{\,\backslash\,} \{ from \} \.{:}}%
\@x{\@s{50.36} \.{\land} checkNetQueue [ chain ] [ to ]}%
\@x{\@s{50.36} \.{\land} Get\_current\_head\_1 ( from ,\, chain ,\, to )}%
\@pvspace{8.0pt}%
\@x{}%
\@y{\@s{0}%
 Request current head from all active peers
}%
\@xx{}%
\@x{ Get\_current\_head\_all \.{\defeq}}%
\@x{\@s{16.4} \E\, from \.{\in} Nodes \.{:}}%
\@x{\@s{27.72} \E\, chain \.{\in} node\_info . active [ from ] \.{:}}%
 \@x{\@s{39.04} \.{\land} network\_info . active [ chain ] \.{\,\backslash\,}
 \{ from \} \.{\neq} \{ \}}%
\@x{\@s{39.04} \.{\land} Get\_current\_head\_n ( from ,\, chain )}%
\@pvspace{8.0pt}%
\@x{}%
\@y{\@s{0}%
 Receive a message
}%
\@xx{}%
\@x{ Receive \.{\defeq}}%
\@x{\@s{16.4} \E\, chain \.{\in} Chains \.{:}}%
 \@x{\@s{27.72} \E\, node \.{\in} network\_info . active [ chain ] \.{:}
 Receive\_msg ( node ,\, chain )}%
\@pvspace{8.0pt}%
\@x{}\midbar\@xx{}%
\@pvspace{8.0pt}%
\@x{}%
\@y{\@s{0}%
 A block is produced on an existing chain
}%
\@xx{}%
\@x{ Block\_production \.{\defeq}}%
\@x{\@s{16.4} \E\, chain \.{\in} 1 \.{\dotdot} network\_info . chains \.{:}}%
 \@x{\@s{27.72} \E\, branch \.{\in} 1 \.{\dotdot} network\_info . branches [
 chain ] ,\, num\_ops \.{\in} 0 \.{\dotdot} N \.{:}}%
\@x{\@s{39.04} Produce\_block ( chain ,\, branch ,\, num\_ops )}%
\@pvspace{8.0pt}%
\@x{}\midbar\@xx{}%
\@pvspace{8.0pt}%
\@x{}%
\@y{\@s{0}%
 Type invariant
}%
\@xx{}%
\@x{ TypeOK \.{\defeq}}%
\@x{\@s{16.4}}%
\@y{\@s{0}%
 active nodes
}%
\@xx{}%
 \@x{\@s{16.4} \.{\land} \A\, node \.{\in} Nodes ,\, chain \.{\in} Chains
 \.{:}}%
 \@x{\@s{35.71} node \.{\in} network\_info . active [ chain ] \.{\equiv} chain
 \.{\in} node\_info . active [ node ]}%
\@x{\@s{16.4} \.{\land} network\_info \.{\in}}%
\@x{\@s{35.71} [ chains\@s{13.17} \.{:} Chains}%
\@x{\@s{35.71} ,\, branches \.{:} [ Chains \.{\rightarrow} Branches ]}%
 \@x{\@s{35.71} ,\, blocks\@s{12.39} \.{:} [ Chains \.{\rightarrow} [ Branches
 \.{\rightarrow} Seq ( Blocks ) ] ]}%
\@x{\@s{35.71} ,\, protocol\@s{3.62} \.{:} Protocols}%
 \@x{\@s{35.71} ,\, active\@s{12.20} \.{:} [ Chains \.{\rightarrow} {\SUBSET}
 Nodes ]}%
 \@x{\@s{35.71} ,\, sent\@s{19.67} \.{:} [ Chains \.{\rightarrow} [ Nodes
 \.{\rightarrow} Seq ( Messages ) ] ]}%
 \@x{\@s{35.71} ,\, recv\@s{20.18} \.{:} [ Chains \.{\rightarrow} [ Nodes
 \.{\rightarrow} Seq ( Messages ) ] ] ]}%
\@x{\@s{16.4} \.{\land} node\_info \.{\in}}%
 \@x{\@s{35.71} [ active\@s{17.45} \.{:} [ Nodes \.{\rightarrow} {\SUBSET}
 Chains ]}%
 \@x{\@s{35.71} ,\, messages \.{:} [ Nodes \.{\rightarrow} [ Chains
 \.{\rightarrow} Seq ( Messages ) ] ]}%
 \@x{\@s{35.71} ,\, branches\@s{1.91} \.{:} [ Nodes \.{\rightarrow} [ Chains
 \.{\rightarrow} {\SUBSET} Branches ] ]}%
 \@x{\@s{35.71} ,\, blocks\@s{14.31} \.{:} [ Nodes \.{\rightarrow} [ Chains
 \.{\rightarrow} [ Branches \.{\rightarrow} Seq ( Blocks ) ] ] ] ]}%
\@pvspace{8.0pt}%
\@x{}\midbar\@xx{}%
\@pvspace{8.0pt}%
\@x{ Init \.{\defeq}}%
\@x{\@s{16.4} \.{\land} network\_info \.{=}}%
\@x{\@s{35.71} [ chains\@s{13.17} \.{\mapsto} 1}%
\@x{\@s{35.71} ,\, branches \.{\mapsto} [ c \.{\in} Chains \.{\mapsto} 0 ]}%
 \@x{\@s{35.71} ,\, blocks\@s{12.39} \.{\mapsto} [ c \.{\in} Chains
 \.{\mapsto} [ b \.{\in} Branches \.{\mapsto} {\langle} {\rangle} ] ]}%
\@x{\@s{35.71} ,\, protocol\@s{3.62} \.{\mapsto} 0}%
 \@x{\@s{35.71} ,\, active\@s{12.20} \.{\mapsto} [ c \.{\in} Chains
 \.{\mapsto} \{ \} ]}%
 \@x{\@s{35.71} ,\, sent\@s{19.67} \.{\mapsto} [ c \.{\in} Chains \.{\mapsto}
 [ n \.{\in} Nodes \.{\mapsto} {\langle} {\rangle} ] ]}%
 \@x{\@s{35.71} ,\, recv\@s{20.18} \.{\mapsto} [ c \.{\in} Chains \.{\mapsto}
 [ n \.{\in} Nodes \.{\mapsto} {\langle} {\rangle} ] ] ]}%
\@x{\@s{16.4} \.{\land} node\_info \.{=}}%
 \@x{\@s{35.71} [ active\@s{17.45} \.{\mapsto} [ n \.{\in} Nodes \.{\mapsto}
 \{ \} ]}%
 \@x{\@s{35.71} ,\, messages \.{\mapsto} [ n \.{\in} Nodes \.{\mapsto} [ c
 \.{\in} Chains \.{\mapsto} {\langle} {\rangle} ] ]}%
 \@x{\@s{35.71} ,\, branches\@s{1.91} \.{\mapsto} [ n \.{\in} Nodes
 \.{\mapsto} [ c \.{\in} Chains \.{\mapsto} \{ \} ] ]}%
 \@x{\@s{35.71} ,\, blocks\@s{14.31} \.{\mapsto} [ n \.{\in} Nodes \.{\mapsto}
 [ c \.{\in} Chains \.{\mapsto} [ b \.{\in} Branches \.{\mapsto} {\langle}
 {\rangle} ] ] ] ]}%
\@pvspace{8.0pt}%
\@x{ Next \.{\defeq}}%
\@x{\@s{16.4} \.{\lor} Activation}%
\@x{\@s{16.4} \.{\lor} Deactivation}%
\@x{\@s{16.4} \.{\lor} Get\_current\_branch\_one}%
\@x{\@s{16.4} \.{\lor} Get\_current\_branch\_all}%
\@x{\@s{16.4} \.{\lor} Get\_current\_head\_one}%
\@x{\@s{16.4} \.{\lor} Get\_current\_head\_all}%
\@x{\@s{16.4} \.{\lor} Receive}%
\@x{\@s{16.4} \.{\lor} Protocol\_upgrade}%
\@x{\@s{16.4}}%
\@y{\@s{0}%
 \ensuremath{TODO} finish
}%
\@xx{}%
\@pvspace{8.0pt}%
\@x{ Spec \.{\defeq} Init \.{\land} {\Box} [ Next ]_{ vars}}%
\@pvspace{8.0pt}%
\@x{}\midbar\@xx{}%
\@pvspace{8.0pt}%
\@x{}%
\@y{\@s{0}%
 Invariants
}%
\@xx{}%
\@x{}%
\@y{\@s{0}%
 \ensuremath{TODO
}}%
\@xx{}%
\@pvspace{8.0pt}%
\@x{}%
\@y{\@s{0}%
 Once a message is sent, it is eventually received by the intended recipient
}%
\@xx{}%
\@x{}%
\@y{\@s{0}%
 A [\ensuremath{msg}] to a [node] ends up in \ensuremath{recv[chain][node]}
 iff \ensuremath{msg \.{\in} in\_messages[chain][node]
}}%
\@xx{}%
\@pvspace{8.0pt}%
\@x{}\bottombar\@xx{}%
\end{document}
